% -------------------- 导言 -------------------- %
\documentclass[UTF8]{ctexart}

% 化学式
\usepackage[version=4]{mhchem}

\usepackage{lmodern}

% 设置页边距
\usepackage{geometry}
\geometry{left=3.18cm, right=3.18cm, top=2.54cm, bottom=2.54cm}

% 图片
\usepackage{graphicx}
% 图片位置
\usepackage{float}
% 并排
\usepackage{subfigure}
\usepackage{parskip}

% 页眉
\usepackage{fancyhdr}
\pagestyle{fancy}
\fancyhead[L]{\small \CJKfontspec{SimHei}第九届生物信息设计与技能竞赛}          % 左页眉
\fancyhead[R]{\small \CJKfontspec{SimHei}大豆根系微生物的基因鉴定与基因功能预测}    % 右页眉
% \fancyhead[L]{中间页眉}
% \fancyfoot[L]{左页脚}
\fancyfoot[C]{第 \thepage 页 \quad 共 \pageref{LastPage} 页}
% \fancyfoot[R]{右页脚}
\renewcommand{\headrulewidth}{1pt} % 分隔线宽度 1 磅
% \renewcommand{\footrulewidth}{4pt}

% 获取最后一页
\usepackage{lastpage}

% 字体
% \usepackage{fontspec}
\setmainfont{Times New Roman} % 英文正文字体


% 设置标号深度
\setcounter{secnumdepth}{3}
\author{}
\date{}

% 设置表格图片标题
\usepackage{caption}
\captionsetup[table]{
    font={small, bf},
    labelsep=quad,
    skip=0pt
}
\captionsetup[figure]{
    font={small, bf},
    labelsep=quad,
    skip=0pt
}

% 摘要
\usepackage{abstract}
% 标题格式 无衬线 加粗 四号
\renewcommand{\abstractnamefont}{\sffamily\bfseries\Large}
% 标题左对齐
\renewcommand{\absnamepos}{flushleft}
% 正文格式
\renewcommand{\abstracttextfont}{\rmfamily\normalsize}
% 正文两端无缩进
\setlength{\absleftindent}{0pt}
\setlength{\absrightindent}{0pt}

% 生成随机文本用于展示排版
\usepackage{lipsum}

% 由于 flushleft/right 取消了 ctex 的默认缩进
% 需要手动设置缩进
\setlength{\parindent}{2em}

% 三线表
\usepackage{booktabs}

% \setlength{\bibleftmargin}{0pt}
% \setlength{bibindent}{2em}

\ctexset{
    % 一级标题 无衬线(黑体),左对齐,四号(14 pt)
    % 标题标号与内容之间空一格
    section/format              +=  \sffamily\raggedright\Large, 
    section/aftername           =   \quad,
    % 二级标题 无衬线(黑体),小四(12 pt) 默认左对齐
    % 标题标号与内容之间空一格
    subsection/format           +=  \sffamily\large,
    subsection/aftername        =   \quad,
    subsection/afterskip        =   0pt,
    % 三级标题 小四(12 pt) 默认衬线 对齐
    % 标题标号与内容之间空一格
    subsubsection/format        +=  \large,
    subsubsection/indent        =   2em,
    subsubsection/number        =   (\arabic{subsubsection}),
    subsubsection/afterskip     =   0pt,
}

\title{\vspace*{-1.5cm} \CJKfontspec{SimHei}大豆根系微生物的基因鉴定与基因功能预测}

% -------------------- 正文 -------------------- %
\begin{document}

    % 令标题页显示页眉页脚
    \maketitle\thispagestyle{fancy}
    \vspace*{-1.5cm}

    % 右对齐 赛道与团队信息
    \begin{flushright}
        {\zihao{5} \heiti  ——竞赛单元“2-1基因组”}

        % 空一行
        \vspace*{\baselineskip} 

        团队名称:xxxx小队 \\
        指导老师:xxx \\
        团队成员:xxx,xxx,xxx,xxx \\
    \end{flushright}


    \begin{abstract}
        请在这里输入摘要\textsuperscript{\cite{ref1}}\lipsum
    \end{abstract}
    

    {\heiti \zihao{-4} \raggedright 关键字:} {\zihao{5} 请在,这里,输入,关键字}


    \section{引言 这是一级标题}

    \subsection{这是二级标题}

    \subsubsection{这是三级标题}

    \lipsum


    % \begin{figure}[!htbp]
    %     \centering
    %     \includegraphics*[width=0.7\textwidth]{img/en_graphlan.pdf}
    %     \caption{培养菌进化分支树(根内)}

    %     \includegraphics*[width=0.7\textwidth]{img/rhi_graphlan.pdf}
    %     \caption{培养菌进化分支树(根际)}
    % \end{figure}



    \section{总结与展望}

    \lipsum

    % 这里是参考文献
    \begin{thebibliography}{0}
        \bibitem{ref1} \lipsum[1]

        \bibitem{ref2} \lipsum[1]

        

    \end{thebibliography}

\end{document}